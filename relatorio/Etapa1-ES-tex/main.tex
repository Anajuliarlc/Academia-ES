\documentclass{article}
\usepackage[utf8]{inputenc}
\usepackage[brazil]{babel}
\usepackage{hyperref}
\usepackage{indentfirst}
\usepackage{xcolor}
\usepackage{amsfonts}
\usepackage{amsmath}
\usepackage{amssymb}
\usepackage{amsthm}
\usepackage{graphicx}
\usepackage{minted}
\usepackage{xurl}
\usepackage[a4paper, left=4cm, right=4cm]{geometry}

\hypersetup{
    colorlinks=true,
    linkcolor=black,
    filecolor=magenta,      
    urlcolor=blue,
    pdftitle={Trabalho ICD},
    pdfpagemode=FullScreen,
    }

\title{Trabalho de Engenharia de Software}
\author{Ana Júlia Rita Lima Cardoso \and George Dutra \and Iago Riveiro Santos Dutra \and Luan Rodrigues de Carvalho \and Luís Henrique Domingues Bueno}
\date{Outubro 2023}

%%%%%%%%%%%%%%%%%%%%%%%%%%%%%%%%%%%%%%%%%%%%%%%%%

\begin{document}

% \maketitle                          
\begin{titlepage}
    \begin{center}
        \vspace*{1cm}
            
        \Huge
        \textbf{Trabalho de Engenharia de Software}
            
        \vspace{0.5cm}
        \LARGE
        Etapa 1
            
        \vspace{1.5cm}
            
        \textbf{Ana Júlia Rita Lima Cardoso, George Dutra, Iago Riveiro Santos Dutra, Luan Rodrigues de Carvalho, Luís Henrique Domingues Bueno}
            
        \vfill
            
        Trabalho desenvolvido e apresentado para\\
        Ciência de Dados e Inteligência Artificial
            
        \vspace{0.8cm}
            
        \includegraphics[width=0.4\textwidth]{Logo FGV.png}
            
        \Large
        Escola de Matemática Aplicada\\
        Fundação Getúlio Vargas\\
        Brasil\\
        Outubro 2023
            
    \end{center}
\end{titlepage}
\newpage

\tableofcontents
\newpage

\section{Casos de Uso e suas descrições}

\subsection{Caso 1: Contatar Personal Trainer}

    \textbf{Fluxo Principal - Caminho Feliz:}

    \begin{itemize}
        \item[1.1 -] Abra o sistema da academia (já logado);
        
        \item[1.2 -] Na área do menu, selecione "Contatar meu Treinador";
        
        \item[1.3 -] Clique na caixa de texto e escreva um resumo do motivo do seu contato;
        
        \item[1.4 -] Será exibida uma mensagem com links de possíveis páginas de FAQ que podem responder sua dúvida, e uma mensagem perguntando se deseja prosseguir com o contato:
        \begin{itemize}
            \item[-] Toque em 'Não' para cancelar o contato (Fim de caso).
            \item[-] Toque em 'Sim' para continuar o contato.
        \end{itemize}
        
        \item[1.5 -]  Selecione uma das opções apresentadas:
        \begin{itemize}
            \item[-] Cancelar contato (Fim de caso).
            \item[-] Conversar com treinador por texto;
            \item[-] Conversar com treinador por voz;
        \end{itemize}
        
        \item[1.6 -] Aguarde até 10 minutos enquanto procuramos por treinadores disponíveis para contato imediato;
        
        \item[1.7 -] O usuário poderá conversar com o treinador por texto ou por chamada em tempo real;
        
        \item[1.8 -] Ao fim do contato, o treinador ou o usuário poderão encerrar a conversa ou a chamada;
        
        \item[1.9 -] Aluno poderá selecionar opção dizendo se sua dúvida foi respondida;
        
        \item[1.10 -] Ambos aluno e treinador poderão avaliar o atendimento, fazer comentários, ou denunciar atitudes um do outro durante o contato que violem as regras da plataforma (Fim de caso).
    \end{itemize}
    
    \textbf{Fluxo Alternativo 1.6 - Treinador não encontrado:}
    
    \begin{itemize}
        \item[1.6.1 -]  Caso não seja encontrado um treinador disponível para atender o aluno, ele poderá:

        \begin{itemize}
            \item[-] Pedir por outra busca (Retorne a 1.5);
            \item[-] Cancelar contato (Fim de caso).
        \end{itemize}
    \end{itemize}

    \textbf{Fluxo Alternativo 1.7 - Usuário inativo:}

    \begin{itemize}
        \item[1.7.1 -] Caso o usuário fique inativo por mais de 15 minutos, o contato poderá ser encerrado pelo treinador.
    \end{itemize}
    \newpage
    \textbf{Fluxo Alternativo 1.7 - Treinador inativo:}

    \begin{itemize}
        \item[1.7.2 -] Caso o professor fique inativo por mais de 15 minutos, o aluno poderá:
        
        \begin{itemize}
             \item[-] Solicitar contato com outro professor (Retorne a 1.5);
             \item[-] Encerrar contato (Avance a 1.9).
        \end{itemize}
    \end{itemize}

    \textbf{Fluxo Alternativo 1.9 - Dúvida não respondida:}

    \begin{itemize}
        \item[1.9.1 -] Caso aluno marque que sua dúvida não foi respondida, ele poderá:

        \begin{itemize}
             \item[-] Buscar por outro treinador (Retorne a 1.5);
             \item[-] Entrar em contato com central de reclamações (Fim de caso);
             \item[-] Encerrar contato (Avance a 1.10).
         \end{itemize}
    \end{itemize}

\subsection{Caso 2: Alterar Cargas de Treino}

    \textbf{Fluxo Principal - Caminho Feliz:}

    \begin{itemize}
        \item[2.1 -] Abra o sistema da academia (já logado);
        \item[2.2 -] Selecione a opção "Meus Treinos" na área do menu;
        \item[2.3 -] Toque na aba "Progressão";
        \item[2.4 -] Selecionando a opção "Progressão Recomendada", o aluno poderá selecionar uma entre algumas recomendações de aumento de carga personalizadas para seu perfil (Fim de caso);
        \item[2.5 -] Selecionando a opção "Progressão Manual", o aluno poderá inserir manualmente as novas cargas que ele passará a usar em cada um de seus treinos (Fim de caso).
    \end{itemize}

    \textbf{Fluxo Alternativo 2.2 - Usuário não possui treino em progresso:}

    \begin{itemize}
        \item[2.2.1 -] Será exibida uma mensagem de erro, na qual o usuário terá a opção de ser redirecionado para a página de contato com treinador ou somente encerrar a operação (Fim de Caso).
    \end{itemize}

    \textbf{Fluxo Alternativo 2.4 - Sem recomendações:}

    \begin{itemize}
        \item[2.4.1 -] Será exibida uma mensagem dizendo que o sistema não tem recomendações de progressão, e recomendará ao usuário duas opções:

        \begin{itemize}
             \item[-] Entrar em contato com um treinador para discutir a progressão, redirecionando à janela de contato com treinadores (Fim de caso);
             \item[-] Inserir progressão manualmente (Retorne a 2.5).
         \end{itemize}
    \end{itemize}

    \textbf{Fluxo Alternativo 2.5 - Carga Inválida:}

    \begin{itemize}
        \item[2.5.1 -] Caso o usuário tente inserir uma carga que não existe, será exibida uma mensagem de erro, e ele será retornado à aba de Progressão Manual (Retorne a 2.5).
    \end{itemize}

    \textbf{Fluxo Alternativo 2.5 - Carga Perigosa:}

    \begin{itemize}
        \item[2.5.2 -] Caso o usuário tente inserir uma carga que é considerada perigosa para seu perfil:

        \begin{itemize}
             \item[-] Será exibida uma mensagem recomendando a ele que fale antes com um Personal;
             \item[-] Serão exibidas três opções:
             \begin{itemize}
                 \item[-] Selecionando "Tenho certeza", a carga será modificada (Fim de Caso);
                 \item[-] Selecionando "Quero contatar um treinador", ele será redirecionado para a janela de contato com um treinador (Fim de caso);
                 \item[-] Selecionando "Cancelar", a operação será encerrada (Fim de Caso);
             \end{itemize}
        \end{itemize}
    \end{itemize}

\subsection{Caso 3: Criar treinos}

    \textbf{Fluxo Principal - Caminho Feliz:}

    \begin{itemize}
        \item[3.1 -] Abra o sistema da academia(já logado);
        \item[3.2 -] Selecione a área de "Treinos" na área do menu;
        \item[3.3 -] Dentro daquele área, clique em "Novo treino";
        \item[3.4 -] Na barra de pesquisa que será aberta, coloque o nome do aluno que deseja criar o treino. Podendo escrever seu nome completo ou escrever sua maior parte e clicar em cima ao ver a opção;
        \item[3.5 -] Após selecionar o aluno, uma nova aba será aberta, na qual será possível escrever o nome do exercício a ser executado, ao seu lado, quantas séries e quantas repetições;
        \item[3.6 -] Clique no "+" abaixo para escrever mais um exercício;
        \item[3.7 -] Ao finalizar, clique em "Finalizar treino";
        \item[3.8 -] Visualize o treino do aluno(Fim de caso).
    \end{itemize}

    \textbf{Fluxo Alternativo 3.5 - Aluno errado:}

    \begin{itemize}
        \item[3.5.1 -] Caso o aluno selecionado para criar treino não seja o desejado:
        
         \begin{itemize}
             \item[-] Clique no botão "Voltar" no canto superior esquerdo da tela;
             \item[-] Retorne ao passo 3.4.
         \end{itemize}
    \end{itemize}

    \textbf{Fluxo Alternativo 3.5 - Deletar exercício:}

    \begin{itemize}
        \item[3.5.2 -] Caso o exercício não seja o desejado ou errado:
        
         \begin{itemize}
             \item[-] Clique no pequeno botão de lixo que fica ao lado do exercício;
             \item[-] Caso deseje acrescentar outro, prossiga para 3.6;
             \item[-] Após terminar, prossiga para 3.7.
         \end{itemize}
    \end{itemize}

    \textbf{Fluxo Alternativo 3.5 - Mudar exercício:}

    \begin{itemize}
        \item[3.5.3 -] Caso tenha digitado o exercício errado ou mudado de ideia, também é possível:
        
         \begin{itemize}
             \item[-] Clique na caixinha que escreveu o exercício e digite o novo exercício;
             \item[-] Prossiga com os passos seguintes.
         \end{itemize}
    \end{itemize}

\subsection{Caso 4: Fazer matrícula}

    \textbf{Fluxo Principal - Caminho Feliz:} \\

    \begin{itemize}
        \item[4.1 -] Abra o sistema da academia(já logado);
        
        \item[4.2 -] Abra a aba "Matrículas" na área do menu;
        
        \item[4.3 -] Clique em "Matricular novo aluno";
        
        \item[4.4 -] Preencha os campos de nome completo, RG, CPF, número de celular e endereço de residência;
        
        \item[4.5 -] Clique em "Prosseguir", que irá guiar ao campo médico;
        
        \item[4.6 -] Preencha a documentação médica, que pergunta se toma algum medicamento, tem algum problema de saúde e pergunta porque decidiu iniciar a academia;
        
        \item[4.7 -] Clique em "Prosseguir", que guiará a área de pagamentos;
        
        \item[4.8 -] Redireciona a "Cadastrar forma de pagamento";
        
        \item[4.9 -] Clique em "Prosseguir", que irá para a tela de confirmação;
        
        \item[4.10 -] Na tela de confirmação aparecerão todas as informações preenchidas durante o processo. Se todas estiverem corretas, clique em prosseguir;
        
        \item[4.11 -] Ao prosseguir, o contrato será exibido, no final haverá um botão de confirmação com os termos do contrato, que deve ser clicado pelo aluno;
        
        \item[4.12 -] Após clicar, a área para assinatura do aluno é disponibilizada e deve ser assinada;
        
        \item[4.13 -] Finalizada a matrícula(Fim de caso).
    \end{itemize}

    \textbf{Fluxo Alternativo 4.10 - Informações básicas erradas:}

    \begin{itemize}
        \item[4.10.1-] Caso as informações da primeira tela de cadastro estejam erradas, a tela de confirmação mostrará (o que guiará o usuário a voltar lá):
        
        \begin{itemize}
            \item[-] Clique em "Voltar a Info Básicas" e repita o passo 4.4;
            \item[-] Prossiga com os passos, que já estarão salvos no sistema.
        \end{itemize}
    \end{itemize}

    \textbf{Fluxo Alternativo 4.10 - Documentação médica errada:}

    \begin{itemize}
        \item[4.10.2-] Caso as informações da documentação médica estejam erradas, a tela de confirmação mostrará(o que guiará o usuário a voltar lá):

        \begin{itemize}
            \item[-] Clique em "Voltar a Doc Médica" e repita o passo 4.6;
            \item[-] Prossiga com os passos, que já estarão salvos no sistema.
        \end{itemize}
    \end{itemize}

    \textbf{Fluxo Alternativo - Voltar:}

    \begin{itemize}
        \item[] Caso digite alguma informação errada e perceba logo após que clicou em "Prosseguir", é possível clicar a qualquer momento no botão que fica no canto superior esquerdo da tela, "Voltar". Permitindo que preencha as informações novamente a qualquer momento.
    \end{itemize}

\subsection{Caso 5: Atualizar Medidas}

    \textbf{Fluxo Principal - Caminho Feliz:}

    \begin{itemize}
        \item[5.1 -] Abra o sistema da academia (já logado);
        \item[5.2 -] Na área do menu, selecione "Progresso";
        \item[5.3 -] Na aba "Progresso" será exibido o histórico de medidas, clique no botão "Atualizar medidas";
        \item[5.4 -] Serão exibidas as últimas medidas registradas, clique na caixa de texto de cada medida e modifique o valor;
        \item[5.5 -] Clique em "Salvar Medidas";
        \item[5.6 -] Visualize a nova medida na aba "Progresso" (Fim do caso).
    \end{itemize}
    
    \textbf{Fluxo Alternativo 5.3 - Medida incorreta:}

    \begin{itemize}
        \item[5.3.1 -] Caso uma medida do histórico esteja incorreta:
        \begin{itemize}
            \item[-] Clique na medida errada;
            \item[-] Selecione uma das opções:
            \begin{itemize}
                \item[-] Deletar medida (Retorne a 5.6).
                \item[-] Atualizar medidas (Retorne a 5.4).
            \end{itemize}
        \end{itemize}
    \end{itemize}
    
\subsection{Caso 6: Cadastrar Novas Aulas}

    \textbf{Fluxo Principal - Caminho Feliz:}

    \begin{itemize}
        \item[6.1 -] Abra o sistema da academia (já logado);
        \item[6.2 -] Na área do menu, selecione "Aulas";
        \item[6.3 -] Na aba "Aulas" serão exibidas todas as aulas e datas registradas, clique em "Nova aula";
        \item[6.4 -] Acrescente o título da aula e a descrição da aula;
        \item[6.5 -] Clique no calendário para exibir os horários;
        \item[6.6 -] Selecione os horários das aulas e sala em que ser[a dada];
        \item[6.7 -] Clique em "Salvar Aulas";
        \item[6.8 -] Confirme com a senha do usuário;
        \item[6.9 -] Visualize a aula na aba "Aulas" (Fim de caso).
    \end{itemize}
    
    \textbf{Fluxo Alternativo 6.6 - Horário e Sala Indisponível:}

    \begin{itemize}
        \item[6.6.1 -] Caso o horário  e sala selecionados já possua uma aula marcada:
        \begin{itemize}
            \item[-] Será exibida com a mensagem "Conflito de Horário" e a aula que está dando conflito;
            \item[-] Selecione uma das opções:
            \begin{itemize}
                \item[-] Alterar horário e manter sala(Retorne a 6.6);
                \item[-] Alterar sala e manter horário(Retorne a 6.6).
            \end{itemize}
        \end{itemize}
    \end{itemize}
    
\subsection{Caso 7: Solicitar troca de treino}

    \textbf{Fluxo Principal - Caminho Feliz:}

    \begin{itemize}
        \item[7.1 -] Abra o sistema da academia (já logado);
        \item[7.2 -] Selecione a aba de "Treinos" na área do menu;
        \item[7.3 -] Dentro da aba, abra o menu kebab do bloco do treino que deseja alterar e selecione "Solicitar Alteração";
        \item[7.4 -] Na caixa de texto que surge, escolha o motivo de troca de treino e escreva o detalhamento do pedido com alguma exigência ou limitação;
        \item[7.5 -] Após revisar a solicitação, clique em "Enviar Solicitação"(Fim de Caso).
    \end{itemize}
    
    \textbf{Fluxo Alternativo 7.4 - Treino errado:}

    \begin{itemize}
        \item[7.4.1 -] Caso o aluno queira que outro treino seja adaptado (e.g. membros superiores, e não inferiores), ele deve:
        \begin{itemize}
            \item[-] Clicar no botão "Voltar" no canto superior esquerdo da tela;
            \item[-] Retornar ao passo 7.3.
        \end{itemize}
    \end{itemize}

    \textbf{Fluxo Alternativo 7.5 - Motivo errado:}

    \begin{itemize}
        \item[7.5.1 -] Caso queira mudar a razão da solicitação:
        \begin{itemize}
            \item[-] Clicar no botão "Voltar" no canto superior esquerdo da tela;
            \item[-] Retornar ao passo 7.3.
        \end{itemize}
    \end{itemize}

    \textbf{Fluxo Alternativo 7.5 - Cancelar solicitação:}

    \begin{itemize}
        \item[7.5.3 -] Caso tenha enviado uma solicitação de maneira equivocada:
        \begin{itemize}
            \item[-] Volte à aba "Treinos" do programa;
            \item[-] Acesse a opção "Minhas solicitações de troca";
            \item[-] Escolha o item que deseja cancelar;
            \item[-] Clique em "Excluir Solicitação".
        \end{itemize}
    \end{itemize}

\subsection{Caso 8: Cadastrar forma de pagamento}

    \textbf{Fluxo Principal - Caminho Feliz:}

    \begin{itemize}
        \item[8.1 -] Abra o sistema da academia (já logado);
        \item[8.2 -] Abra a aba "Perfil" na área do menu;
        \item[8.3 -] Clique em "Informações de Pagamento";
        \item[8.4 -] Clique em "Cadastrar Novo Meio de Pagamento";
        \item[8.5 -] Escolha o tipo de meio de pagamento;
        \item[8.6 -] Se o meio de pagamento escolhido for cartão de crédito, preencher as informações do cartão;
        \item[8.7 -] Clique em "Prosseguir", que irá para a tela de confirmação;
        \item[8.8 -] Na tela de confirmação aparecerão todas as informações preenchidas durante o processo. Se todas estiverem corretas, clique em confirmar(Fim de Caso).
    \end{itemize}

    \textbf{Fluxo Alternativo 8.6 - Cartão de débito:}

    \begin{itemize}
        \item[8.6.1 -] Caso o aluno escolha pagar com cartão de débito:
        \begin{itemize}
            \item[-] Preencher as informações do cartão;
            \item[-] Prosseguir ao passo 8.7.
        \end{itemize}
    \end{itemize}

    \textbf{Fluxo Alternativo 8.6 - Débito automático em conta corrente:}

    \begin{itemize}
        \item[8.6.2 -] Caso o aluno escolha pagar com débito automático:
        \begin{itemize}
            \item[-] Preencher informações da conta corrente;
            \item[-] Prosseguir ao passo 8.7
        \end{itemize}
    \end{itemize}

    \textbf{Fluxo Alternativo 8.6 - Boleto recorrente:}

    \begin{itemize}
        \item[8.6.3 -] Caso o aluno escolha pagar com boleto recorrente:
        \begin{itemize}
            \item[-] Prosseguir ao passo 8.7.
        \end{itemize}
    \end{itemize}

    \textbf{Fluxo Alternativo 8.6 - Informações incongruentes:}

    \begin{itemize}
        \item[8.6.4 -] Caso a API do banco não reconheça as informações inseridas.
        \begin{itemize}
            \item[-] Corrigir informação(ões) incongruente(s), a(s) qual(is) o sistema identificará por uma mensagem de erro em fonte vermelha abaixo da(s) respectiva(s) caixa(s) de input.
        \end{itemize}
    \end{itemize}

\subsection{Caso 9: Marcar aulas coletivas}

    \textbf{Fluxo Principal - Caminho Feliz:}

    \begin{itemize}
        \item[9.1 -] Abra o sistema da academia (já logado);
        \item[9.2 -] Seleciona "Aulas" na área do menu;
        \item[9.3 -] Seleciona "Marcar Aulas";
        \item[9.4 -] Selecione uma aula disponível dentre as apresentadas;
        \item[9.5 -] O aluno reserva o dia e horário dentre os apresentados para a aula;
        \item[9.6 -] O sistema mostra uma confirmação na sua tela de que você se inscreveu numa aula(Fim de Caso);
    \end{itemize}
    \newpage
    \textbf{Fluxo Alternativo 9.4 - Número de Alunos máx. Atingido:}

    \begin{itemize}
        \item[9.4.1 -] Caso o número máximo de alunos numa aula tenha sido atingido:
        \begin{itemize}
            \item[-] Ao selecionar a aula naquele dia e horário o sistema abrirá uma aba que dirá que a aula está cheia e que o aluno pode tentar selecionar outra data;
            \item[-] Repita o passo 9.4 e caso nenhum dia estiver possível para aquela aula, repita o passo 9.3.
        \end{itemize}
    \end{itemize}

    \textbf{Fluxo Alternativo 9.2 - Cancelar Aulas:}

    \begin{itemize}
        \item[9.2.1 -] Caso o aluno deseje cancelar sua inscrição numa aula:
        \begin{itemize}
            \item[-] Volte ao ponto 9.2, que mostrará todas as suas aulas marcadas;
            \item[-] Selecione a aula que deseja cancelar;
            \item[-] Clique em "Cancelar";
            \item[-] Uma aba de confirmação de cancelação aparecerá na sua tela e a aula em questão sumirá da lista de aulas inscritas do aluno.
        \end{itemize}
    \end{itemize}

\subsection{Caso 10: Definir metas}

    \textbf{Fluxo Principal - Caminho Feliz:}

    \begin{itemize}
        \item[10.1 -] Abra o sistema da academia (já logado);
        \item[10.2 -] Seleciona "Progresso" na área do menu;
        \item[10.3 -] O aluno seleciona "Definir metas";
        \item[10.4 -] O aluno preenche os campos de metas de treino, tais como minutos de corrida diária, meta de perda de peso/calorias, ou ganho de massa muscular;
        \item[10.5 -] O aluno salva as metas criadas no sistema;
        \item[10.6 -] Todas as metas aparecerão numa aba de visualização para o aluno verificar(Fim de Caso).
    \end{itemize}

    \textbf{Fluxo Alternativo 10.2 - Cancelar Metas:}

    \begin{itemize}
        \item[10.2.1 -]  Caso o aluno deseje cancelar todas as suas metas:
        \begin{itemize}
            \item[-] No passo 10.2 haverá o botão "Cancelar Metas", selecione-o;
            \item[-] Ao selecionar, uma aba de confirmação será aberta, perguntando se tem certeza do processo;
            \item[-] Caso tenha, clique em "Sim".
        \end{itemize}
    \end{itemize}

    \textbf{Fluxo Alternativo 10.2 - Editar Metas:}

    \begin{itemize}
        \item[10.2.2 -] Caso o aluno deseje editar suas metas:
        \begin{itemize}
            \item[-] No passo 10.2 haverá o botão "Editar Metas", selecione-o;
            \item[-] Ao selecionar, uma aba de confirmação será aberta, perguntando se tem certeza do processo;
            \item[-] Caso tenha, clique em "Sim";
            \item[-] A aba de edição será aberta com os antigos dados salvos, o aluno poderá editar suas metas aqui;
            \item[-] Quando terminar, clique em "Salvar".
        \end{itemize}
    \end{itemize}

\section{Estórias de usuário: }

\subsection{Estória 1: }
    Contatar Personal:  Como um aluno da academia, eu gostaria de poder contatar um personal trainer a qualquer momento, em horário comercial, por mensagem ou ligação, para tirar dúvidas, pedir dicas, ou fazer reclamações a respeito do meu treinamento atual.
\subsection{Estória 2: }
    Alterar cargas: Como aluno, preciso ser capaz de atualizar as cargas de peso de treino no aplicativo da academia para acompanhar meu progresso e atualizar o nível de dificuldade dos meus exercícios.
\subsection{Estória 3: }
    Criar Treinos: Como professor, é preciso poder criar os treinos dos alunos para acompanharem os métodos adequados a eles.
\subsection{Estória 4: }
    Fazer Matrícula: Como professor, preciso fazer a matrícula dos alunos,  colocando suas limitações e preocupações, para que o contrato com a academia seja adequado às suas necessidades. 
\subsection{Estória 5: }
    Atualizar Medidas: Como aluno, gostaria de atualizar minhas medidas pessoais (peso, medidas e etc.) para acompanhar meu progresso nas metas estabelecidas.
\subsection{Estória 6: }
    Criar Aulas: Como personal trainer, preciso criar novas aulas (boxe, dança, etc.) para disponibilizar a todos os alunos as aulas fornecidas pela academia.
\subsection{Estória 7: }
    Solicitar troca de treino: Como um aluno, gostaria de solicitar mudanças em meu plano de treinamento para meu professor, para assim poder evoluir ou adaptá-lo a uma lesão.
\subsection{Estória 8: }
    Cadastrar forma de pagamento: Como aluno, gostaria de poder cadastrar uma nova forma de pagamento para pagar a academia e para não precisar solicitar isso ao atendente em caso de bloqueio de cartão.
\subsection{Estória 9: }
    Marcar aulas coletivas: Como um aluno, preciso ser capaz de reservar uma aula para garantir minha participação e uso dos recursos disponíveis da academia.
\subsection{Estória 10: }
    Definir metas: Como um aluno, preciso ser capaz de definir minhas metas de treino diárias, semanais ou mensais, para que eu possa acompanhar meu progresso.

\section{Metáfora do Sistema: }

O sistema virtual de uma academia pode ser comparado a uma "Cozinha de Restaurante Fitness". Nesta metáfora, o sistema virtual assume o papel dos chefes e cozinheiros que preparam pratos saudáveis para os clientes, explicando mais sobre o assunto, podemos dividir em algumas categorias, sendo elas:

Chefes Virtuais: O sistema virtual é como o chef de cozinha do restaurante em questão. Ele é o encarregado de criar receitas de exercícios, planos de treinamento e rotinas personalizadas para cada cliente, no caso, os alunos.

Cozinheiros Virtuais: Os "cozinheiros" representam os diferentes componentes do sistema virtual, como programas de treinamento, metas estabelecidas e aulas oferecidas, além de também podermos compreender como os personais da academia virtual. Eles trabalham juntos para fornecer um serviço completo aos clientes.

Clientes Famintos por Saúde: Os alunos da academia são como os clientes famintos por saúde e condicionamento físico. Eles chegam à "cozinha" em busca de refeições de exercícios e aulas bem preparadas para atender às suas necessidades específicas.

Cardápio de Exercícios: O cardápio da academia é composto por exercícios variados e programas de treinamento. Os chefes e cozinheiros virtuais ajudam os clientes a escolher os pratos (exercícios) adequados para seus objetivos. Mais além, podemos entender que as metas criadas pelos alunos sejam suas comidas favoritas e nossa cozinha fará o possível para servi-la da melhor forma possível.

Pratos Personalizados: Assim como um chef personaliza pratos com base nas preferências do cliente, o sistema virtual personaliza rotinas de exercícios com base nas metas e capacidades de cada cliente como citado anteriormente, levando em questão, o peso, medida, tudo relacionado a metas e progresso do cliente.

Serviço de Mesa e Feedback: Os garçons e atendentes representam o sistema virtual que fornece feedback e orientação aos clientes durante seus treinos. Eles se asseguram de que os pratos sejam servidos da maneira correta e de que os clientes tenham uma experiência satisfatória.

Qualidade e Sabor: A qualidade dos exercícios e a eficácia do treinamento são comparáveis ao sabor dos pratos em um restaurante. O sistema virtual garante que os exercícios e aulas sejam eficazes e agradáveis para cada cliente, levando em consideração suas limitações e preocupações, que até poderiam ser interpretadas como alergias e desgostos.

Nesta metáfora, a academia é vista como um restaurante fitness onde os clientes recebem pratos personalizados de exercícios preparados por chefes e cozinheiros virtuais. O sistema virtual desempenha o papel central na criação de experiências de treinamento deliciosas e saudáveis para os frequentadores famintos por condicionamento físico.
    
\end{document}
